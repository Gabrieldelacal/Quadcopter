%%%%%%%%%%%%%%%%%%%%%%%%%%%%%%%%%%%%%%%%%%%%%%%%%%%%%%%%%%%%%%%%%%%%%%%%%%%
%
% Plantilla para un tabajo en LaTeX en español.
%
%%%%%%%%%%%%%%%%%%%%%%%%%%%%%%%%%%%%%%%%%%%%%%%%%%%%%%%%%%%%%%%%%%%%%%%%%%%

\documentclass{article}

% Esto es para poder escribir acentos directamente:
\usepackage[utf8]{inputenc}
% Esto es para que el LaTeX sepa que el texto está en español:
\usepackage[spanish]{babel}

% Paquetes para escribir ecuaciones matemáticas
\usepackage{amsmath, amsthm, amsfonts}


% Atajos.
% Se pueden definir comandos nuevos para acortar cosas que se usan
% frecuentemente. Como ejemplo, aquí se definen la R y la Z dobles que
% suelen representar a los conjuntos de números reales y enteros.
%--------------------------------------------------------------------------

\def\RR{\mathbb{R}}
\def\ZZ{\mathbb{Z}}

% De la misma forma se pueden definir comandos con argumentos. Por
% ejemplo, aquí definimos un comando para escribir el valor absoluto
% de algo más fácilmente.
%--------------------------------------------------------------------------
\newcommand{\abs}[1]{\left\vert#1\right\vert}

% Operadores.
% Los operadores nuevos deben definirse como tales para que aparezcan
% correctamente. Como ejemplo definimos en jacobiano:
%--------------------------------------------------------------------------
\DeclareMathOperator{\Jac}{Jac}
\textwidth=17cm
\textheight=24cm
\topmargin=-1cm
\oddsidemargin=-1.5cm

%--------------------------------------------------------------------------
\title{Da Quadcopter Vinci}
\author{Joan Guasch Iglesias\\Gabriel de la Cal Mendoza\\\\
  \small Tecnologia de Control\\
}

\begin{document}


%Esto fuerza al compilador a hacer el título
\maketitle


%esta orden fuerza un salto de pàgina
\newpage


%Nosotros no tenemos derecho a escoger el formato de los títulos, Latex lo hace por nosotros
\section{Introducción}
Con este proyecto se pretende entender cómo se ejecuta control de un quadcopter, y aplicarlo. 

\section{Objetivo}
Se plantean para este trabajo varios objetivos de dificultad creciente, conforme se vayan cumpliendo todos los anteriores. Nos consideraremos aprobados si se logra alcanzar el segundo objetivo de la lista:
\begin{itemize}
	\item Estabilizar el quadcopter para que se mantenga estable en una rótula esférica. Para ello se usa la junta que usualmente se encuentra en una mopa para limpiar el suelo.
	\item Estabilizar el quadcopter para que se mantenga estable suspendido en el aire. Para ello se usará un sensor infrarojo para mantener una distancia constante con el suelo.
	\item Manteniéndose estable en el vuelo, debe ser éste capaz de esquivar a objetos que se le acerquen en ciertas direcciones en el plano.
	\item Controlar por RC al quadcopter mientras esquiva los objetos que se encuentre en el camino manteniéndose estable.
\end{itemize}

\section{Contexto de trabajo} 
Este trabajo es un proyecto ya empezado en la asociación AESS, y la intención es retomar este reto para acabar con el proyecto "Quadcopter" tanto por su atractiva temática como por el espacio que ocupa en la estantería.
Está todo el proyecto basado en el Kid ArduCopter \cite{1}, una plataforma donde ya se han predefinido las librerías adecuadas para manejar las funciones básicas. El hardware, también preparado del mismo Kid, es de la plataforma libre Arduino.
%Si hay algún tipo de referencia es bueno explicarlo aqui mediante una referencia como esta ,\cite{Cd95}

\section{Desarrollo}
Las fases que se pretenden seguir para desarrollar el proyecto están cada una ligada a un objetivo del trabajo, en el que conlleva cada uno un reto nuevo y no contemplado hasta el momento. Para el control del autómata en una rótula esférica se ha pensado que el control deba ser Proporcional y Diferencial, porque interesa aproximarse a la consigna de manera asintótica. Al tratarse de un sistema mecánico las inercias son grandes así como las constantes de tiempo, por lo que las reacciones rápidas para corregir el error no son viables. Y como no se quieren overshoots tampoco se podrá aplicar el método de Ziegler-Nichols.
%¿Cual es vuestro plan de desarrollo?
%Explicar un poco las fases del proyecto

\subsection{Software} 

En el kid del que se parte ya están incluidas librerías para la obtención de los datos y la emisión de órdenes. Se usan las acciones de estas librerías.

\subsection{Hardware} 

Se usa la placa ArduPilotMega V1 para el Quadcopter. Para ello se deben ensamblar todos los componentes de la placa \cite{2}.

\subsection{Integración}
La comunicación entre el software y el hardware esta integrada en la placa. Para el control de los motores se hará mediante PWM(Modulación de Potencia).

\section{Tests}
En la práctica de cada prueba para verificar que se cumplen los objetivos del sistema se procede de distinta manera: 
\begin{itemize}
	\item Para el primer objetivo se mantendrá al Quadcopter en una bancada, representada por una unión de una mopa de limpiar a modo de rótula esférica. 
	\item En el caso de estar probando la estabilidad del vuelo, se cuelga al sistema de una cuerda para evitar accidentes y "simular" una libertad de movimientos. También se usará este sistema para comprobar el resto de objetivos.
\end{itemize}
%Queremos que especifiqueis los experiemntos que se pueden %hacer para verificar que habeis alcanzado los objetivos que %se habian marcado al inicio del proyecto


% Bibliografía.
%-----------------------------------------------------------------
\begin{thebibliography}{2}

\bibitem{1}    http://code.google.com/p/arducopter/wiki/ArduCopter,(última visita en 29/03/2012)
\bibitem{2}
http://code.google.com/p/arducopter/wiki/AC2Assembly, (última visita en 29/03/2012)
%\bibitem{Cd95} Autor, \emph{Título}, Revista/Editor, (año)
\end{thebibliography}

\end{document}
